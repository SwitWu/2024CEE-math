\documentclass[noamssymb]{ctexbeamer}
\usepackage{lete-sans-math}
\usepackage{array}
\newcolumntype{C}{>{$}c<{$}}
\setcounter{MaxMatrixCols}{24}
\newcommand{\red}{\color{red}}
\newcommand{\blu}{\color{blue}}
\begin{document}

\begin{frame}{2024 年新课标 I 卷压轴题}
  设 $m$ 为正整数,数列 $a_1$, $a_2$, $\ldots$, $a_{4m+2}$ 是公差不为 $0$ 的等差数列,
  若从中删去两项 $a_i$ 和 $a_j$ $(i < j)$后,剩余的 $4m$ 项可被平均分为 $m$ 组,
  且每组的 $4$ 个数都能构成等差数列,则称数列 $a_1$, $a_2$, $\ldots$, $a_{4m+2}$
  是 $(i, j)$-可分数列.
  \begin{enumerate}[(1)]
  \item 写出所有的 $(i,j)$, $(1 \leq i < j \leq 6)$,
    使数列 $a_1,a_2,\ldots,a_6$ 是 $(i, j)$-可分数列;
  \item 当 $m \geq 3$ 时,证明数列 $a_1$, $a_2$, $\ldots$, $a_{4m+2}$ 是 $(2, 13)$-可分数列;
  \item 从 $1,2,\ldots, 4m+2$中一次任取两个数 $i$ 和 $j$ $(i < j)$,
    记数列 $a_1$, $a_2$, $\ldots$, $a_{4m+2}$ 是 $(i, j)$-可分数列的概率是 $P_{m}$,
    证明:$P_{m} > \frac{1}{8}$
  \end{enumerate}
\end{frame}

\begin{frame}
  \footnotesize
  \setlength{\tabcolsep}{2pt}
  \begin{tabular}{C<{\hspace{8pt}}CCCCCCCCCCCCCCCCCCCCCC}
    m=1 & 1 & 2 & 3 & 4 & 5 & 6 \\
    m=2 & \color<2->{red}1        & \color<2->{lightgray}2 
        & \color<2->{red}3        & \color<2->{blue}4
        & \color<2->{red}5        & \color<2->{blue}6 
        & \color<2->{red}7        & \color<2->{blue}8
        & \color<2->{lightgray}9  & \color<2->{blue}10 \\
    m=3 & \color<3->{red}1        & \color<3->{lightgray}2 
        & \color<3->{blue}3       & \color<3->{red}4 
        & \color<3->{purple}5     & \color<3->{blue}6 
        & \color<3->{red}7        & \color<3->{purple}8 
        & \color<3->{blue}9       & \color<3->{red}10 
        & \color<3->{purple}11    & \color<3->{blue}12 
        & \color<3->{lightgray}13 & \color<3->{purple}14 \\
    m=4 & \color<4->{red}1        & \color<4->{lightgray}2 
        & \color<4->{blue}3       & \color<4->{purple}4 
        & \color<4->{red}5        & \color<4->{cyan}6 
        & \color<4->{blue}7       & \color<4->{purple}8 
        & \color<4->{red}9        & \color<4->{cyan}10 
        & \color<4->{blue}11      & \color<4->{purple}12 
        & \color<4->{red}13       & \color<4->{cyan}14 
        & \color<4->{blue}15      & \color<4->{purple}16 
        & \color<4->{lightgray}17 & \color<4->{cyan}18 \\
    m=5 & \color<5->{red}1        & \color<5->{lightgray}2 
        & \color<5->{blue}3       & \color<5->{purple}4 
        & \color<5->{cyan}5       & \color<5->{red}6 
        & \color<5->{green}7      & \color<5->{blue}8 
        & \color<5->{purple}9     & \color<5->{cyan}10 
        & \color<5->{red}11       & \color<5->{green}12 
        & \color<5->{blue}13      & \color<5->{purple}14 
        & \color<5->{cyan}15      & \color<5->{red}16 
        & \color<5->{green}17     & \color<5->{blue}18 
        & \color<5->{purple}19    & \color<5->{cyan}20 
        & \color<5->{lightgray}21 & \color<5->{green}22 \\
  \end{tabular}
\end{frame}


\end{document}